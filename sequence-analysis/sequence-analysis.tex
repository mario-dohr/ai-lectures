\documentclass[10pt,a4paper]{article}
\usepackage[utf8]{inputenc}
\usepackage{amsmath}
\usepackage{amsfonts}
\usepackage{amssymb}
\usepackage{xcolor}
\author{Mario Dohr}
\title{Sequence Analysis}
\begin{document}
\maketitle

\section{Molecular Biology Intro}
The \textbf{Procaryotic} cell has no nucleus. Chromosome holds the DNA. Second source of DNA is the plasmid.

The \textbf{Eurakyotic} cell has a \textbf{Nucleus}. The nucleus holds the DNA. Inside the nucleus is the \textbf{Nucleolus} which is important in producing the \textbf{Ribosomes} which are making proteins out of the \textbf{mRNA}. Eukaryotes also have \textbf{Mitochondria} which have their own DNA and ribosomes.
The ribosomes that produces the proteins are sitting in the \textbf{Endoplasmic Reticulum}

The \textbf{Golgi Aparatus} adresses where the proteins belongs to, some have to go to the cell wall. He is the distributer of the proteins of the cells.

The \textbf{Plant Cell} have a additional structure the \textbf{chloroplast} and the \textbf{vacuole} that can store food and water is bigger.

\subsection{DNA and RNA}

DNA is made of 4 bases: \textbf{Adenine} and \textbf{Guanine} which are called \textbf{Purines}. The two other bases are \textbf{Cytosine} and \textbf{Thymine} the are called \textbf{Pyrimidine}.

\paragraph{Building the (double) chain}
The bases of one chain are connected to a sugar which connect itself via a phosphate group to the next sugar.
Hydro bonds are connecting the bases of the two strands together. A can only pair with T and C can only pair with G. \textcolor{red}{Hydrogen bonds and covalent bonds?}

\paragraph{Base Pairing in DNA}
The possible pairings are A-T and G-C. The pairing between G-C has 3 hydro bonds and the pairing between A-T has two hydro bonds. So G-C pairing is more stable and in nature we can see that organisms that live in harsh condition typically have a higher G-C content.

\paragraph{Double Helix}
Double stranded DNA winds up itself into the famous double helix.
\textcolor{red}{5' vs 3'?}


\paragraph{Chromosomes}
\textcolor{red}{chromosome?}
DNA is segmented into chromosomes in eurkaryotes only. Each organisms has its typical chromosomal structure. Chromosomes are numberd from largest to smallest. Images of chromosomal structure are called \textbf{karyotypes}
Each chromosome has has a \textbf{Centromere} which divides the chromosome into the p arm and the q arm.
The \textbf{Telomeres} are at the end of the chromosomes.

No two genomes are identical (except twins). if $ > 1 \% $  of a population has the same variation in the same position, we speak of a \textbf{Single Nucleotide Polymorphism} or \textbf{SNP}.

\paragraph{DNA vs RNA}
DNA und RNA bestehen aus Nukleotiden. Ein Nukleotid hat drei Zutaten:
\begin{itemize}
\item Zuckermolekül
\item Phosphatgruppe
\item organisch, stickstoffhaltige Base von denen es 4 (5) gibt.
\end{itemize}
Mehrere verknüpfte Nukleotide ergeben eine Kette (=Polynukleotid), die abwechselnd aus Zucker und Phosphatgruppe (-Zucker-Phosphat-Zucker-Phosphat-) besteht. 
An einem Ende befindet sich der Phosphatrest (5' Ende), am anderen hingegen die OH- Gruppe (3'-Ende). Die DNA/RNA-Kette besitzt also immer eine bestimmte Richtung oder Polarität und zwar immer vom 5'Ende zum 3'Ende ( 5' -> 3').

\paragraph{Differences in the structure}
\begin{itemize}
\item DNA = Deoxyribonucleic acid, RNA = Ribonucleic acid. $\rightarrow$ DNA = RNA - oxygen.
\item DNA hase the base Thymin RNA has Uracil.
\item RNA is single stranded DNA is double stranded.
\item RNA is short (view hundred bases). DNA is long (a view millions). 
\item DNA is stable, RNA is unstable.
\end{itemize}

\paragraph{Differences in function}
\begin{itemize}
\item DNA stores Erbinformationen. Diese werden an nachfolgende Generationen weitergegeben. Replikation.

\end{itemize}

DNA stores information, RNA transports information.

\paragraph{Types of RNA}
mRNA is messenger RNA, tRNA is transfer RNA, Ribosomal RNA rRNA, regulatory RNA.

\subsection{Amino acids, proteins}
\textbf{Proteins} are the workhorses in the cell. They are involved in practically every proces in the cell. They have catalytic activity, structural function (hair), and many others.

\textbf{Amino acids?}.

\paragraph{Replication}
First unwinding the DNA. 

\paragraph{Transcription}
Bedeutet umschreiben. Die Transkription ist der erste Schritt der Proteinbiosynthese.
Die Informationen der DNA werden in eine mRNA umgeschrieben, diese wird dann zu den Riposomen transportiert wo die Proteine hergestellt werden.

\textbf{Schritte in der Transcription}
\begin{itemize}
\item Initiation: a protein complex places RNA-polymerase to promoter sequence.
\item Elongation: RNA-P moves along the DNA, unwinding, copying (=create RNA), re-winding
\item Termination: RNA-P stops and releases RNA.
\end{itemize}

\paragraph{Translation}
Create proteins out of the mRNA.
\end{document}