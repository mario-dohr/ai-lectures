\documentclass[10pt,a4paper]{article}
\usepackage[utf8]{inputenc}
\usepackage{amsmath}
\usepackage{amsfonts}
\usepackage{amssymb}
\author{Mario Dohr}
\title{Do artefacts have politics?}
\begin{document}
\maketitle
\section{Deutsche Zusammenfassung}
- Maschinen und Technik nicht nur nach Effizienz etc beurteilen, sondern auch wie sich Macht und Authorität verkörpern können.

- Laut Mumford immer zwei Arten von Technologien nebeneinander: eine autoritäre (systemzentriert, mächtig, instabil) und demokratisch (menschenzentriert, schwach aber einfallsreich und dauerhaft)

- Vermehrte Kernkraft führt zu autoritären System, da nur sie die Sicherheit garantieren können.

- Kaum technische Errungenschaft die nicht von jemanden als die Rettung der freien Gesellschaft bezeichnet wurde. 
Technischer Fortschritt als Garant für soziale Gleichheit, Freiheit und Demokratie.

- Technische Errungschaften, der technologische Fortschritt, die Industrialsierung sind natürlich tief mit den Bedingungen der modernen Politik verwoben. Daraus zu Argumentieren, das bestimmte Technologien politische Eigenschaften haben scheint auf den ersten Blick falsch, da Politik immer noch von Menschen und nicht von Maschinen gemacht wird. Daher auch der Einwand an jene, die dies dennoch befürworten: Es kommt nicht auf die Technik an, sondern darauf, wie sie in das soziale und wirtschaftliche System eingebettet ist. Das nennt man soziale Determination der Technik. Diese Sichtweise ist ein Gegenpol zum naiven technischen Deterministmus - die Vorstellung, dass sich Technik nur aus internen Dynamik entwickelt und dann unabhängig von jedem anderen Einfluss , die Gesellschaft formt. 
Aber auch der soziale Determinismus hat Mängel, da er suggeriert, das Technik keine Rolle spielt: Wenn man die Ursprünge und Machthaber die hinter einem technischen Wandel stehen aufgedeckt hat, dann hat man alle Wichtige erklärt. 

- die Theorie der technologischen Politik lenkt die Aufmerksamkeit auf die Eigendynamik soziotechnischer Großsysteme, auf die Reaktion von Gesellschaften auf technologische Imperative, auf die Anpassung menschlicher Ziel an technische Mittel

- Im folgenden zwei Arten, wie Artefakte politische Eigenschaften enthalten können. In einem Fall wird das Design, die Anordnung etc. bewusst gewählt um eine bestimmte Auswirkung zu haben. Im anderen Fall gibt es bestimmte inhärent politische Technologien, das sind Technologien, die bestimmte Arten von politischen Beziehungen zu erfordern scheinen.

1. Technische Vorkehrungen als Formen und Ordnung.
- Brücken in Long Island
- McCormik pneumatische Formmaschine (diese zwei waren beabsichtigt)
- Benachteiligung von Behinderten (unbeabsichtig)
Zusammengefasst:Technologien beeinflussen die Ordnung von sozialen Systemen. Diese Technologien bieten viele Möglichkeiten und wir entscheiden uns bewusst oder unbewusst für eine. Dies beeinflusst dann für Generationen wie wir kommunizieren, reisen, arbeiten. Wenn Strukturentscheidungen getroffen werden, haben verschieden Menschen unterschiedlichen Einfluss. Der größte Freiraum besteht, wenn eine Technik zum ersten mal eingeführt wird. Sobald jedoch die ersten Verpflichtungen eingegangen wurden verschwindet diese Flexiblität. In diesem Sinne sind technologische Innovationen vergleichbar mit Gesetzen.

2. Inhärent politische Technologien




\end{document}