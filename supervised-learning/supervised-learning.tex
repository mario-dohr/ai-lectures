\documentclass[10pt,a4paper]{article}
\usepackage[utf8]{inputenc}
\usepackage[english]{babel}
\usepackage{amsmath}
\usepackage{amsfonts}
\usepackage{amssymb}
\usepackage{graphicx}
\author{Mario Dohr}
\title{Supervised Learning}
\begin{document}
\maketitle
\section{Introduction}
\begin{itemize}
\item Learning a function that maps an input to an output (=target).
\item Learning is based on example input values with corresponding target values.
\item Typical usage: train model on dataset with input + target values and then use the model to predict target values for new inputs.
\item Two types: Classification and Regression
\end{itemize}

\textbf{Terminology}
\begin{itemize}
\item Model: parameterized function with specific parameter values
\item Model class: the class of models in which we search for the model (e.g. neural network, SVM, etc)
\item Parameters representation of concrete models inside the given model class
\item Hyperparameters: paramters controlling the model complexity. Not learned. E.g. number of neurons in NN)
\end{itemize}

Generalization Error/Risk
\$ R(g(.;w))\$
\end{document}